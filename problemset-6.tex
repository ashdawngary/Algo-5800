\documentclass[11pt, a4paper]{article}

% One of the following is required: problemset, recitation, quiz, exam
% The following are required: handoutnum, assigneddate.
% If there is only one date, set both duedate and assigneddate to be the same.
% Do not change handoutnum or dates
\usepackage[
problemset,
handoutnum=6,
assigneddate={20 November 2021},duedate={30 November 2021},
% % Uncomment the line below IF these are solutions
% solution,
% % Uncomment the line below IF you are a student submitting solutions
% student,
% % Replace with your name
name={Fill Submitter's Name},
% Replace with names of all group members who collarborated on this.
% If unsure about ordering, then you can follow the convention in theory
% and list names alphabetically by last name.
groupmembers={Fill collaborators' names}
]{course-handouts-preamble}

% Be careful of commas and put text with spaces within {curly braces}
% Don't use a comma at the end, but do use commas between options.
% Weird errors occur otherwise, I wasted some time failing to debug those.
% DO NOT EDIT
% These are fixed values that should not be changed during this course.
\pgfkeys{/chp/fixed/.cd,
instructorname = {Ravi Sundaram},
coursename = {CS5800 Algorithms}}


%
%% START Preamble: Reductions via Restrictions
\newcommand{\clique}{\textsc{Clique}}
\newcommand{\subgraphIsomorphism}{\textsc{Subgraph Isomorphism}}
\newcommand{\partition}{\textsc{Partition}}
\newcommand{\multiprocessorScheduling}{\textsc{Multiprocessor Scheduling}}
%
%% END Preamble: Reductions via Restrictions
%
% Add any macros you want below, or put them in a separate file and \input{file}
% keeping the preamble clean can keep you sane.

\begin{document}
% Do not change either of the below lines.
\insertHandoutInfoBox{}
\ifbool{isexam}{\input{exam-blurb}} %comment this line only if it throws an error.

% Start adding content from below here.


\newproblem{Reducing 2-coloring to SAT}{15}

\begin{problem}[\textsf{2-Coloring}]
  Given an undirected graph $G=(V, E)$, where the graph has $|V|$ nodes and $|E|$ edges, is there a way to color the nodes $V$ either black or white such that for every edge, it has two different colors on its endpoints?
\end{problem}
\begin{problem}[\textsc{SAT}]
  Given $n$ variables $x_1, x_2, \dots, x_n$, and $m$ clauses $C_1, C_2, \dots, C_m$ with each clause is an OR between some literals $z_i$ and each literal is either a variable or the negation of a variable, does there exist an assignment of True/False to the variables such that $C_1 \land C_2 \land \dots \land C_m$ is True?
\end{problem}
\begin{enumerate}[label=\alph*.]
\item Reduce the \textsc{2-Coloring} problem to \textsc{SAT}. That is, write a polynomial sized set of clauses (need not have 3 literals each) using a polynomial number of variables such that if there is a satisfying assignment if and only if there is a \textsc{2-Coloring}. Briefly describe what the variables and clauses represent.

  \hint{Note that  $(x_1 \lor x_2) \land (\neg x_1 \lor \neg x_2)$ is satisfied when exactly one of $x_1, x_2$ are true. Use this fact to enforce the coloring constraint.} \\ \\
 Let us consider a graph $G = (V,E)$.  For each $v \in V$, we can create a variable $v$ that is true if it is colored black, and false if it is colored white. \\ \\
 Next for each edge $e = (s,t) \in E$, we need to assert that either $(s \lor t) \land (\lnot s \lor \lnot t)$, ie exactly one of s or t (the endpoints are black).  It then suffices to chain each of these clauses with ands for each edge.  \\ \\
 The 2-COLORING problem reduces to SAT with $2m$ clauses which is poly(m), thus this is a polynomial reduction from 2coloring to sat. \\ \\
 \textbf{Correctness}: Lets assume there is an assignment to the SAT problem.  Then for each variable, we can assign black or white depending on the truth value of the solution that satisfies it, and by the construction of our clauses, each edge will contain exactly 1 black and white endpoint. As a result 2Coloring $\leq_p$ SAT.
 
\item Does this reduction imply that \textsc{2-Coloring} is NP-Complete? Why? \\ 
\textbf{No}.  We demonstrated that 2Coloring $\leq_p$ SAT, which mean that 2Color is at most as hard as SAT.   \\ \\
2Color can be solved in polynomial time (hence it cannot be NP complete, unless $P=NP$), but in order to show that it is NP complete, we would need to demonstrate that SAT $\leq_p$ 2Color, namely every instance of SAT can be poly reduced to 2 color.  
\end{enumerate}


\newproblem{Reductions via Restrictions}{10+10}

A common strategy to prove NP-Completeness of new problems is to show that problems already known to be NP-Complete are in fact special cases of the new problem. This is often referred to as the ``restriction'' strategy, where the new problem is restricted to be equivalent to a known NP-Complete problem. Use the ``restriction'' strategy to prove the following new problems to be NP-Complete by reducing from the given known NP-Complete problem. You only need to provide the specific manner in which these are restrictions along with a brief description of correctness.

\begin{enumerate}
\item Give a poly-time reduction to \subgraphIsomorphism{} from \clique.
\begin{problem}[\subgraphIsomorphism]
  Given two graphs $G=(V_G, E_G)$ and $H= (V_H, E_H)$, does $G$ contain a subgraph that is isomorphic to $H$? That is, is there a subset $V\subseteq V_G$ and a subset $E\subseteq E_G$ such that $|V| = |V_H|, |E|=|E_H|$, and there exists a one-to-one function $f: V_H \to V$ satisfying $\set{u, v} \in E_h$ if and only if $\set{f(u), f(v)} \in E$?
\end{problem}

\begin{problem}[\clique]
  Given a graph $G= (V,E)$ and a positive integer $k\leq |V|$, does $G$ contains a clique $V'$ having $|V'| \geq k$? That is, is there a subset $V'\subseteq V$ such that, for every pair of vertices $u, v \in V'$, the edge $\{u, v\} \in E$.
\end{problem} 
Consider a problem in clique with graph $G = (V,E)$, and some $k$.  First Observe that clique of $k+1$ $\to$ clique of $k$ as we can strip away any node within the clique and the property is maintained. It follow that is suffices to find any clique f size $k$. \\ \\
Secondly observe that every clique is an isomorphism to the complete graph of size $k$, $K_k$.  By definition of clique, every pair of nodes must be directly connected, which is the exact description of a complete graph. \\ \\
\textbf{Reduction}: It suffices to template match a graph of $K_k$ (complete graph of size $k$) onto our original graph.  We can pass this into subgraph isomogrphism as our $H$, and our original graph $G$, to see if there is some subset $E \subseteq E_G$ such that it is a isomorphism to our $H$, $K_k$. \\ \\
\textbf{Correctness}: Since every clique is a isomorphism of $K_k$, and isomorphisms are transitive, it follows that such subgraph extracted by SUBGRAPH ISOMORPHISM must be isomorphic to clique.  In order to recover the answer, we can apply the mapping function to each node and edge in $K_k$, to obtain a subgraph of $G$.  Should there exist no subgraph, in G, there is no clique either. \\ \\ \textbf{Poly Reduction}: This reduction is polynomial as it suffices to construct $H$ which is a function of $K$.  There are $O(K^2)$ edges to construct and $O(K)$ nodes to introduce for the graph, which is poly k.

\item Give a poly-time reduction to \multiprocessorScheduling{} from \partition.
\begin{problem}[\multiprocessorScheduling]
  Given a finite set $A$ of ``tasks'', a positive integer ``length'' $\ell(a)$ for each ``task'' $a\in A$, a positive integer $m$ of ``processors'', and a positive integer ``deadline'' $D$, is there a partition (of the ``tasks'' into the $m$ ``processors'') $A = A_1 \cup A_2 \cup \dots \cup A_m$ of $A$ into $m$ disjoint sets such that each ``processor'' finishes all of its ``tasks'' by the ``deadline'':
  \begin{align*}
    \max_{1\leq i\leq m}\left(  \sum_{a\in A_i}\ell(a)   \right) \leq D?
  \end{align*}
\end{problem}

\begin{problem}[\partition]
  Given a set $A$, a positive integral size function $s(a)$ for all elements $a\in A$, is there a partition of $A$ into two subsets $A_1, A_2$ so that the sum of sizes in $A_1$ is equal to that in $A_2$, that is, $\sum_{i\in A_1}s(i) = \sum_{j\in A_2}s(j)$?
\end{problem}
\end{enumerate} \\ 

Reduction PARTITION to MPS.  In PARITTION, it suffices to find a partition of the set $S$, (A,B) such that sum(A) = sum(B).  Observe that since the sum is constant, $sum(B) = sum(S)-sum(B) \iff 2sum(B) = sum(S)$. \\ \\
This reveals that PARITION only has a solution if the sum S is even.  Seconly, it suffices to manipulate the second equation as 
$$sum(B) = sum(S) - sum(B)$$
$$sum(B) = \frac{1}{2} sum(S) + \frac{1}{2} sum(S) - sum(B)$$
$$sum(B) - \frac{1}{2} sum(S) = \frac{1}{2}sum(S) - sum(B)$$.
Since LHS is the negative of RHS, it follows that any parition of the tasks will either yield one set below $\frac{1}{2}sum(S)$ and one above, \textbf{unless} there is a valid paritition in which they are both equal.  This motivates the following redunction:  \\ \\
\textbf{Reduction}: Given a problem in parititon $S$, if $sum(S)$ is odd, we can map it to FALSE( any problem in MPS that yields false) \\ 
if $sum(S)$ is even, we can map it to MPS with 2 processors where each element in $S$ has a equivalent job with length of the elements value, and a deadline of $\frac{1}{2}sum(S)$.  \\ \\
Since it suffices to transform each element of the set into a job in MPS, it follows that this reduction is linear in the size of $S$, which is polynomial of the original problem.  \\ \\
The correctness of the algorithm is justified in the motivation, showing that equality between the two sizes only holds if and only if they both sum to $\frac{1}{2} sum(S)$.



\newproblem{$K$-weight path in a DAG}{10+15+5}

In class you learned that if we can reduce a NP-complete problem $A$ to problem $B$ through a polynomial time reduction, then problem $B$ must be at least as hard as $A$, or NP-hard\footnote{To show NP-Completeness, you must further show that $B$ is in NP, that is, given a solution to $B$, we can verify whether the solution is correct in polynomial time.}. In this problem, we'll use a reduction to show that a problem is NP-hard using a reduction from an NP-complete problem.

The NP-complete problem you are given is \textsc{Subset Sum}, which you have seen (or will see) in class. In the \textsc{Subset Sum} problem you are given an input array $A$ of $n$ positive integers and an integer $T$, and you must decide whether there exists a subset of indicies in $\{1,...,n\}$ such that the sum of $A$ of these indices is exactly $T$. That is, given $A$, $n$, and $T$ you should return true if and only if
\[\exists S \subseteq \{1,...,n\}\text{ such that }\sum_{s \in S}A[s] = T.\]

We wish to prove that the \textsc{$K$-Path} problem is NP-hard using a reduction from \textsc{Subset Sum}. In the \textsc{$K$-Path} problem you are given a DAG (directed acyclic graph) $G = (V,E)$ with non-negative edge weights $w:E \rightarrow \mathbb{N}$ and a pair of vertices $s$ and $t$ in $V$, and you must decide whether there exists an $s$-$t$ path in $G$ whose total weight is $K$. That is, given $G$, $w$, $s$, $t$, and $K$ you should return true if and only if \[\exists \text{ an $s$-$t$ path $P$ in $G$ such that }\sum_{e \in E(P)}w(e) = K.\]

Show that the \textsc{$K$-Path} problem is NP-hard using a reduction from \textsc{Subset Sum}. That is, given an instance $SS = (A,n,T)$ for \textsc{Subset Sum}, show how to transform that into an instance $KP = (G,w,s,t,K)$ for \textsc{$K$-Path} such that \textsc{Subset Sum} returns true on $SS$ if and only if \textsc{$K$-Path} returns true on $KP$.

\begin{enumerate}[label=\alph*)]
\item Show how to polynomially reduce an instance of \textsc{Subset Sum} into an instance of \textsc{$K$-Path}. Make sure that your output is a valid instance of \textsc{$K$-Path} (i.e. all weights are non-negative and $s$ and $t$ are well-defined). \\ \\
At every point in subset sum, we either have the decision to include an element or disclude it.  This motivates us to construct a directed acyclic graph as follows. 
\begin{itemize}
    \item Create nodes $a_i$ for $i = 0, 1, \cdots, n$
    \item Let $s$ be $a_0$ and $t$ be $a_n$. 
    \item For each element $A[i]$ ($1 \leq i \leq n$), attach an edge of weight 0 between $a_{i-1}$ to $a_i$, and another edge of weight $A[i]$ between  $a_{i-1}$ to $a_i$.
\end{itemize}
It suffices to solve KPATH for the new graph between $s$ and $t$ with a target value of $T$.

\item Show that your reduction is correct. That is, show that \textsc{Subset Sum} returns true on an input instance to your reduction if and only if \textsc{$K$-Path} returns true on the output of your reduction.

Let us assume there does exist a path in our augmented graph.  It suffices to reconstruct the subset that would also sum to $T$. \\ \\
Given the path that satisfies the KPATH problem posed above, it suffices to include an element if its edge was taken. More specifically, between every pair of consecutive nodes, the final path either goes through the 0 edge, or the nonzero edge.  It suffices to include every element for which the solution goes through the nonzero edge.  Since we add up the nonzero edge sums in our path, and it equals T, it follows that the subset sum of the corresponding subset must also be T. \\ \\
In the case where there is no such path in the graph, it folows that any path in the graph will not have pathsum T.  Since there is a bijection between every subset and every path in the graph we created, it follows that no subset in our original problem has sum T.

\item Show that your reduction takes polynomial time.

In order to create the graph, we create 2 edges per element in the set.  Therefore the size of the graph is linear in the size of the set, which is polynomial of the size of the input.  Therefore it is a polynomial reduction, and takes polynomial time to generate.
\end{enumerate}


\newproblem{Camp Recruiting Problem}{5+10+15+5}

Suppose you're helping to organize a summer sports camp, and the following problem comes up. The camp covers a set $S=\set{\text{baseball, volleyball, }\dots}$ of $n$ sports. For each of the $n$ sports in $S$, the camp is supposed to have at least two counselors who are skilled in that sport. The camp has received a set $C$ of $m$ job applications from potential counselors. For each sports $s\in S$, there is some subset $A_s\subseteq C$ of the counselor applicants qualified in that sport.

We define the Efficient Recruiting problem as: For a given number $1 \leq k \leq m$, is it possible to hire at most $k$ of the counselors and have at least $2$ counselors qualified in each sports in $S$? More formally, given a number $1 \leq k \leq m$, is there a subset $H\subseteq C$, with $|H|=k$ such that for all $s\in S$, $A_s \cap H \geq 2$ (every sports has at least two counselors hired who are skilled in it).

Show that the Efficient Recruiting problem is NP-Complete by reducing from Set Cover. In particular show each of the following:

\begin{enumerate}[label=\alph*)]
\item Show that Efficient Recruiting is in NP. \\ \\
It suffices to detail a nondeterministic polynomial time algorithm: \\ \\
Given a set $S$ of sports, $C$ candidates, a size $k$, and $A_1, A_2, \cdots, A_n$ satisfiable subsets we can: 
\begin{itemize}
    \item \textbf{nondeterminsitically} branch of all $k$-subsets of $C$ 
    \item for each subset $H$ 
        \begin{itemize}
        \item for each satisfaction set $A_i$ ($i \in 1, \cdots n$)
        \begin{itemize}
            \item Count the number of elements of $H$ that are in $A_i$.  \textbf{reject} if less than 2.
        \end{itemize}
        \item \textbf{accept} (return true).
      \end{itemize}
    \item suffices to \textbf{accept} as long as any thread accepts, else \textbf{reject}
\end{itemize}

\textbf{Algorithm is in NP}: For each thread, it suffices to check if $H$, a $k$-subset of $C$ satifies the conditions.  This takes $O(km^2)$ which is polynomial in the input.  Therefore Efficient Recruiting is in NP.

\item Show how to polynomially reduce an instance of Set Cover into an instance of Efficient Recruiting. Make sure that your output is a valid instance of Efficient Recruiting.

Let $S_1, S_2, \cdots, S_n$ be the set of subsets we are working with.  We want to obtain a union of $U$ with at most $k$ subsets. \\ \\
For each elemnt of $U$, we construct a sport for it. We will also introduce a counselor $d$ such that $d \in A_i, \forall i \in U$. This relaxes Efficent Recruiting to return true as long as there is one couneselor who satisfies a sport (since $d$ can serve every sport). \\ \\
Next for each subset $S_i, \forall i \in 1, \cdots, K$, we create counselor $i$.  The counselor is in $A_x \forall x \in S_i$.  \\ \\
Finally it suffices to run Efficient Recruiting on $\{d\} \cup \{1, \cdots, n\}$, with constraints $A$, sports $U$, and counselor cap $k + 1$. \\ \\
\textbf{k+1 rationale}: In order to reduce the 2-counselor requirement to 1, $d$ must be selected (it is always optimal to include $d$ in any soluition, detailed in proof of correctness), which means that we need to include 1 extra slot, as we want to select $k$ of the original subsets. \\ \\

\item Show that your reduction is correct. That is, show that Set Covers returns true on an input instance to your reduction if and only if Efficient Recruiting returns true on the output of your reduction. \\ \\

Lets assume there some subset of counselors within ER which satisfies the constraints( $S$).  Either
\begin{itemize}
    \item $d \in S$, in which it suffices to take $S - \{d\}$, which is a size $k$ set.  These counselors are mapped to subsets which we can select, and their union must be the universe (as $d$ contributed once to each sport, which is bijected to the universe).
    \item $d \not \in S$, in which it suffices to remove any arbitrary element of the set and include $d$.  This is because subset $d$ is $U$, which is the superset of any subset( of $U$), hence removing any subset and adding back d will produce a valid solution $\iff$ the original solution was valid. \\ \\
    Since $d \in S$, it suffices to follow the procedure from the bullet above.
\end{itemize}

Following simmilar logic, if there is no k+1 subset of counselors, then since $d$ covers all, there is no k subset that unions to $U$. 
\item Show that your reduction takes polynomial time.
It suffices to show that the transformation from Set Cover to ER is polynomial.  \\ \\
In order to construct A, it suffices to iterate over the contents of every subset.  If there are $m$ subsets originally and $n = |U|$, it follows that we must make $O(mn)$ assignments (to transform contents of subsets to constraints on counselors), and it follows that this is polynomial in the size of the problem provided.
\end{enumerate}


\end{document}